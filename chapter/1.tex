\section{ Chapter 1 }

\bigskip\bigskip\bigskip

\begin{question}{}{}
If $r$ is rational ($r\neq0$) and $x$ is irrational,
prove that $r+x$ and $rx$ are irrational.
\end{question}
\begin{Proof}
Suppose $r+x\in\mathbf{Q}$.

Since $r\in\mathbf{Q}$  and $r+x\in\mathbf{Q}$, 
thus $r+x-r=x\in\mathbf{Q}$.

$x\in\mathbf{Q}$,
contradicting the fact $x$ is irrational.

Thus $r+x$ is irrational.\bigskip

Suppose $rx\in\mathbf{Q}$.

Similarly, since $r\in\mathbf{Q}$ and $rx\in\mathbf{Q}$,
thus $\frac{rx}{r}=x\in\mathbf{Q}$.

$x\in\mathbf{Q}$,
contradicting the fact $x$ is irrational.

Thus $rx$ is irrational.

\hfill$Q.E.D.$\qedhere\hspace{42pt}
\end{Proof}
\begin{question}{}{}
Prove that there is no rational number whose square is 12.
\end{question}
\begin{Proof}
Suppose $r\in\mathbf{Q}$ and $r^2=12$,
where $r=\frac{m}{n}\; (m,n\in\mathbf{Z},\:n\neq0)$, 
and $m,n$ are relatively primes.

So, $m^2=12n^2$,
since $3|12n^2$, thus $3|m^2$.

Thus $3|m$, thereby $9|m^2$.

Noting that $m^2=12n^2$, and $9|m^2$.

Thus $3|4n^2$, i.e. $3|n^2$, i.e. $3|n$,
contradicting the fact $n$ and $m$ are relatively primes.

So $\nexists r\in\mathbf{Q},\;r^2=12$.

\hfill$Q.E.D.$\qedhere\hspace{42pt}

\end{Proof}
\begin{question}{}{}
Prove Proposition 1.15.
\end{question}
\begin{proposition}{1.15}{}
\itshape{
The axioms for multiplication imply the following statements. 

(a) If $x\neq0$ and $xy=xz$ then $y=z$. 

(b) If $x\neq0$ and $xy=x$ then $y=1$. 

(c) If $x\neq0$ and $xy=1$ then $y=1/x$. 

(d) If $x\neq0$ then $1/(1/x)=x$.
}
\end{proposition}
\begin{Proof}
Since $x\neq0$, $\exists x^{-1}$ is the inverse of $x$. 

If $xy=xz$, it gives:
\begin{align*}
y &=1\times y=(x^{-1} x)y=x^{-1}(xy) \\
  &=x^{-1}(xz)=(x^{-1}x)z=z
\end{align*}

This proves (a). Take $z=1$ in (a) to obtain (b).

Take $z=x^{-1}$ in (a) to obtain (c).

Since $x^{-1}x=1$, (c)(with $x^{-1}$ in place of $x$) gives (d).

\hfill$Q.E.D.$\qedhere\hspace{42pt}
\end{Proof}
\begin{question}{}{}
Let $E$ be a nonempty subset of an ordered set;
suppose $\alpha$ is a lower bound of $E$ 
and $\beta$ is an upper bound of $E$.
Prove that $\alpha\le\beta$.
\end{question}
\begin{Proof}
Since $E$ is nonempty, and $\alpha$ is a lower bound of $E$.

Thus $\forall x\in E,\;x\ge\alpha$.

Similarly, $\forall x\in E,\;x\le\beta$.

Thus $\forall x\in E,\;\alpha\le x\le\beta$.
i.e. $\alpha\le\beta$.

\hfill$Q.E.D.$\qedhere\hspace{42pt}
\end{Proof}
\begin{question}{}{}
Let $A$ be a nonempty set of real numbers which is bounded below.
Let $-A$ be the set of all numbers $-x$, where $x\in A$.
Prove that 
$$\inf A=-\sup(-A).$$
\end{question}
\begin{Proof}
Let $\alpha=\inf A$, $\forall x\in A$, $\alpha\le x$

Thus, $\forall x\in A,\:-x\le-\alpha$. i.e. let $y=-x,\:\forall y\in-A,\:y\le-\alpha$. 

which means $-\alpha$ is the upper bound of $-A$.

Since $\alpha=\inf A$, thus $\forall x>\alpha$,
$x$ is not the lower bound of $A$.

Thus $\forall-x<-\alpha$, $-x=y$ is not the upper bound of $-A$.

Since $-\alpha$ is the upper bound of $-A$,
and $\forall y<-\alpha$, $y$ is not the upper bound of $-A$.

Thus $-\alpha=\sup(-A)$, i.e. $\inf A=-\sup(-A)$.

\hfill$Q.E.D.$\qedhere\hspace{42pt}
\end{Proof}
\begin{question}{}{}
Fix $b>1$. 

(a) If $m,\:n,\:p,\:q$ are integers, $n>0,\:q>0$,
and $r=m/n=p/q$, prove that$$(b^m)^{1/n}=(b^p)^{1/q}$$

Hence it makes sense to define $b^r=(b^m)^{1/n}$.

(b) Prove that $b^{r+s}=b^rb^s$ if $r$ and $s$ are rational. 

(c) If $x$ is real,
define $B(x)$ to be the set of all numbers $b^t$,
where $t$ is rational and $t\le x$.
Prove that $$b^r=\sup B(r)$$
when $r$ is rational.
Hence it makes sense to define $$b^x=\sup B(x)$$
for every real $x$.

(d) Prove that $b^{x+y}=b^xb^y$ for all real $x$ and $y$.
\end{question}
\begin{Proof}
(a)\quad Let $x=(b^m)^{1/n}$,
which means $x^n=b^m$.

Thus $x^{nq}=b^{mq}$, i.e. $x^q=(x^{mq})^{1/n}$.

Since $m/n=p/q$, gets $p=qm/n$,
hence $x^q=(b^{mq})^{1/n}=b^p$.
i.e. $x=(b^p)^{1/n}$.

Thus, $(b^m)^{1/n}=(b^p)^{1/q}$.
\newline

(b)\quad Let $r=\frac{m}{n},\:s=\frac{x}{y}$,
so $r+s=\frac{m}{n}+\frac{x}{y}=\frac{my+nx}{ny}$

Thus,
$b^{r+s}=b^\frac{my+nx}{ny}=(b^{my}b^{nx})^{\frac{1}{ny}}=(b^m)^{\frac{1}{n}}(b^x)^{\frac{1}{y}}=b^rb^s$
\newline

(c)\quad Let $x=r$,
then $\forall t,\:t\le r$, when $b^t\in B(r)$, $b^t\le b^r$,
$b^r$ is the upper bound of $B(r)$.

$\forall x<r\;\;\exists t,\:x<t<r,\:b^t>b^x$,
which means the $x$ is not the upper bound of $B(r)$.

Thus $b^r=\sup B(r)$ holds.
\newline

(d)\quad
To prove $b^{x+y}=b^xb^y$,
we can prove $\sup B(x+y)=\sup B(x)\sup B(y)$.

Now to prove $\sup B(x)\sup B(y)$ is the upper bound of $B(x+y)$.

$\forall r\in\mathbf{Q},\:r\le x+y$,
since $\exists r_1\le x,\:r_2\le y,\:r_1+r_2=r$,
which gives
$b^{r_1}\le\sup B(x),\:b^{r_2}\le\sup B(y)$,
i.e. $b^{r_1+r_2}\le\sup B(x)\sup B(y)$,
i.e. $b^{r}\le\sup B(x)\sup B(y)$.

Then to prove $\sup B(x)\sup B(y)=\sup B(x+y)$.

Suppose $\exists z$ is the upper bound of $B(x+y)$,
and $z<\sup B(x)\sup B(y)$.

Since $z$ is the upper bound of $B(x+y)$.

$\forall r_1\le x,\:r_2\le y,\:b^{r_1}b^{r_2}\le z$ and $\sup B(x)b^{r_2}\le z,\:b^{r_1}\sup B(y)\le z$.

So $\sup B(x)b^{r_2}b^{r_1}\sup B(y)\le zz<z\sup B(x)\sup B(y)$,
i.e.$b^{r_1}b^{r_2}<z$,

contradicting with $z\le b^{r_1}b^{r_2}$.

Thus, there are no upper bound of $B(x+y)$ less than $\sup B(x)\sup B(y)$.

Done.

\hfill$Q.E.D.$\qedhere\hspace{42pt}
\end{Proof}
\newpage